\documentclass[12pt, a4paper, oneside]{ctexart}
\usepackage{amsmath, amsthm, amssymb, graphicx}
\usepackage[bookmarks=true, colorlinks, citecolor=blue, linkcolor=black]{hyperref}
\usepackage{geometry}
\geometry{left=2.54cm, right=2.54cm, top=3.18cm, bottom=3.18cm}
\linespread{1.5}
\newtheorem{theorem}{定理}[section]
\newtheorem{definition}[theorem]{定义}
\newtheorem{lemma}[theorem]{引理}
\newtheorem{corollary}[theorem]{推论}
\newtheorem{example}[theorem]{例}
\newtheorem{proposition}[theorem]{命题}
% 导言区
\title{Lecture7}
\author{罗淦}
\date{\today}
\begin{document}
\maketitle
\tableofcontents
\newpage
% 正文
净收益率(相对利润),没有股息收入的前提下:
$R_t=\frac{P_t-P_{t-1}}{P_{t-1}}$

连续复合收益率(对数收益率)
$$r_t=\ln(1+R_t)=\ln(\frac{P_t}{P_{t-1}})=\ln(P_t)-\ln(P_{t-1})=p_t-p_{t-1}$$
$p_t=ln(P_t)$是对数价格

多期对数收益率是单期对数收益率之和
$r_t(k), R_t(k)$是$t-k$时刻到$t$时刻的对数收益率和简单收益率
$$r_t(k)=\ln(1+R_t(k))=\ln()$$







\end{document}